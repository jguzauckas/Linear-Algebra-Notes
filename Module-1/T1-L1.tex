\documentclass{article}

\title{Module 1 \textendash{} Linear Systems and Span \\
        Topic 1 \textendash{} Systems of Linear Equations \\
        Lesson 1 \textendash{} Solution Sets of Linear Equations}
\author{John Guzauckas}
\date{05/16/2022}

\begin{document}
\maketitle

\section{Topics}
We will explore the following concepts:

\begin{itemize}
    \item Systems of Linear Equations
    \item Elementary Row Operations
\end{itemize}

\section{Learning Objectives:}
Students should be able to do the following after watching the video and
completing the assigned homework:

\begin{itemize}
    \item Apply elementary row operations to solve systems of linear equations.
\end{itemize}

\section{A Single Linear Equation}
A linear equation has the form
\[a_{1}x_{1} + a_{2}x_{2} + \cdots + a_{n}x_{n} = b\]
$a_{1}$, $a_{2}$, $\dots$, $a_{n}$ and $b$ are the \textbf{coefficients}, 
$x_{1}$, $x_{2}$, $\dots$, and $x_{n}$ are the \textbf{variables}, 
and $n$ is the \textbf{dimension}, or number of variables.

For example:

\begin{itemize}
    \item $2x_{1} + 4x_{2} = 4$ is a line in 2 dimensions
    \item $3x_{1} + 2x_{2} + x_{3} = 6$ is a plane in 3 dimensions
\end{itemize}

\section{Systems of Linear Equations}

\section{Two Variable Case}

\section{Three Variable Case}

\section{Row Reduction by Elementary Row Operations}

\section{Summary}

\section{Practice 1}

\section{Practice 2}
\end{document}